\documentclass[conference]{IEEEtran}
\IEEEoverridecommandlockouts
% The preceding line is only needed to identify funding in the first footnote. If that is unneeded, please comment it out.
\usepackage{cite}
\usepackage{amsmath,amssymb,amsfonts}
\usepackage{algorithmic}
\usepackage{graphicx}
\usepackage{textcomp}
\usepackage{xcolor}
\def\BibTeX{{\rm B\kern-.05em{\sc i\kern-.025em b}\kern-.08em
    T\kern-.1667em\lower.7ex\hbox{E}\kern-.125emX}}
\begin{document}

\title{House Price Prediction \\Using Regression Algorithms\\
{\footnotesize \textsuperscript{}}
\thanks{Identify applicable funding agency here. If none, delete this.}
}

\author{\IEEEauthorblockN{1\textsuperscript{st} Md. Jusef}
\IEEEauthorblockA{\textit{Dept. of CSE} \\
\textit{East West University}\\
2017-2-60-160@std.ewubd.edu}
\and
\IEEEauthorblockN{2\textsuperscript{nd} Sk. Amir Hamza}
\IEEEauthorblockA{\textit{Dept. of CSE} \\
\textit{East West University}\\
2017-1-60-091@std.ewubd.edu}
\and
\IEEEauthorblockN{3\textsuperscript{rd} Tanim Hasan Mahmud}
\IEEEauthorblockA{\textit{Dept. of CSE} \\
\textit{East West University}\\
2017-1-60-130@std.ewubd.edu}
}

\maketitle

\begin{abstract}
people are cautious when they are attempting to purchase a new house with their budgets and advertise methodologies. The objective of the paper is to figure the coherent house costs for non-house holders based on their financial conditions and their goals. The paper includes expectations utilizing distinctive Regression techniques like Linear Regression, Random forrest Regression, Decision tree Regression. House cost expectation on a data set has been done by utilizing all the over mentioned techniques to discover out the finest among them. The motive of this paper is to help the seller to estimate the selling cost of a house perfectly and to help people to predict the exact time slap to accumulate a house. A few of the related variables that affect the cost were moreover taken into thought such as physical conditions, concept and area etc.
\end{abstract}

\begin{IEEEkeywords}
house price prediction,linear regression,random forest,decession tree
\end{IEEEkeywords}

\section{Introduction}
The most inspiration of the project forecasting varieties on house price was to make the most excellent thinkable forecast of house costs by using appropriate calculations and finding out which among them is best appropriate for anticipating the cost with low error rate. This is an curiously issue since most of the individuals will eventually buy/sell a house. This problem allows us to learn more about the housing market and helps with making more informed decisions. The analysis that were done in this paper is basically based on the Boston housing data. In this paper we try to use suitable Regression techniques for solving the issue. We used Random Forrest Regression, Linear Regression and Decision Tree Regression and calculate the error rate. 

\section{Dataset}

\subsection{Features Description}

In this study, I used Boston Housing dataset. The dataset comprises 13 input features and 1 target feature. The Boston Data frame has 506 rows and 14 columns.
Attribute information are given below:\\
\\
1. CRIM      per capita crime rate by town\\
2. ZN           proportion of residential land        zoned for lots over 25,000\\ sq.ft.\\
3. INDUS     proportion of non-retail business acres per town\\
4. CHAS      Charles River dummy variable (= 1 if tract bounds  river; 0 otherwise)\\
5. NOX       nitric oxides concentration (parts per 10 million)\\
6. RM        average number of rooms per dwelling\\
7. AGE       proportion of owner-occupied units built prior to 1940\\
8. DIS       weighted distances to five Boston employment centres\\
9. RAD       index of accessibility to radial highways\\
10. TAX      full-value property-tax rate per 10,000 doller\\
 
11. PTRATIO  pupil-teacher ratio by town\\
12. B        1000(Bk - 0.63) power 2 where Bk is the proportion of blacks\\ by town\\
13. LSTAT    percent lower status of the population\\
14. MEDV     Median value of owner-occupied homes in 1000's doller\\

